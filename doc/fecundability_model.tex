\documentclass{article}

% Language setting
% Replace `english' with e.g. `spanish' to change the document language
\usepackage[english]{babel}


% Set page size and margins
% Replace `letterpaper' with `a4paper' for UK/EU standard size
\usepackage[a4paper, top=2cm,bottom=2cm,left=3cm,right=3cm,marginparwidth=1.75cm]{geometry}

% Useful packages
\usepackage{amsmath,amsfonts}
\usepackage{graphicx}
\usepackage{authblk}
\usepackage{float}
\usepackage[colorlinks=true, allcolors=blue]{hyperref}
\usepackage{natbib}
\usepackage{indentfirst}
\usepackage{amsmath}
% \usepackage{palatino}

\title{A Mathematical model for Human Fecundability}

\author[1+,*]{Arjun Biddanda}
\author[1]{Sara A. Carioscia}
\author[1]{Rajiv C. McCoy}
\affil[1]{Department of Biology, Johns Hopkins University}


\begin{document}
\maketitle

\section*{Model Description}

We briefly provide an introduction to the model below, with the initial summarized probabilities under consideration. The  probability we would like to model is the probability of a **live birth conditional on conception occurring**. We model this using a series of conditional probabilities: 

\begin{equation}
	P(\text{Live Birth}) =  (1 - P(\text{EPL} | \text{implantation})) \cdot (1 - P(\text{failed implantation}))
\end{equation}
This model posits that a health live birth (conditional on conception occurring) is conditional on both no implantation failure **and** no early pregnancy loss. While there may be a number of potential factors influencing both of these negative events during pregnancy, we primarily rely on the assumption that aneuploidy is a primary driver of these effects. We first posit that the distribution of aneuploidy outcomes come from the following distribution: 

\begin{equation}
\begin{aligned}
P(\text{Aneuploidy} = a) &= \begin{cases}
0.4 &, a = \text{meiotic}\\
0.05 &, a = \text{tripolar mitotic}\\
0.1 &, a = \text{mosaic}_{low}\\
0.05 &, a = \text{mosaic}_{high}\\
0.4 &, a = \text{euploid}\\
\end{cases}
\end{aligned}
\end{equation}



These are approximate figures but as long as the proportions sum to 1 here all of the following aspects of the model will follow as needed.  We can then model the implantation failure rate directly: 

$$
P(\text{failed implantation}) = P(\text{failed implantation} | \text{Aneuploidy}) = \begin{cases}
0.8 &, a = \text{meiotic}\\
1.0 &, a = \text{tripolar mitotic}\\
0.4 &, a = \text{mosaic}_{low}\\
0.5 &, a = \text{mosaic}_{high}\\
0.05 &, a = \text{euploid}\\
\end{cases}
$$

This conditional probability suggests that different forms of aneuploidy have different impacts on implantation - roughly corresponding to when they may arise during development (e.g. meiotic-origin being the earliest). We can then compute the final relevant probability distribution as the following: 

$$
P(EPL | \text{implantation}, \text{Aneuploidy}=a) = \begin{cases}
1 - \eta &, a = \text{meiotic}\\
1 - \eta &, a = \text{tripolar mitotic}\\
0.2 &, a = \text{mosaic}_{low}\\
0.5 &, a = \text{mosaic}_{high}\\
\epsilon &, a = \text{euploid}\\
\end{cases}
$$

where $\eta$ is a kind of "escape" probability that is very small where a meiotic or tripolar aneuploidy could not lead to an early-pregnancy loss and $\epsilon$ is the probability of other lethal genetic mutations that may lead to aberrant pregnancy outcomes. 

In particular,  $\epsilon$ is the key parameter of the number of lethal mutations that occur in a euploid embryo and can still lead to pregnancy loss. 



\section*{Results}





\bibliographystyle{plainnat}
\bibliography{refs}

\end{document}
